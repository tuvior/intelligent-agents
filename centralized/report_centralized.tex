\documentclass[11pt]{article}

\usepackage{amsmath}
\usepackage{textcomp}

% Add other packages here %


% Put your group number and names in the author field %
\title{\bf Excercise 4\\ Implementing a centralized agent}
\author{Group \textnumero 1: Tobias Bordenca, Paul Nicolet}


% N.B.: The report should not be longer than 3 pages %


\begin{document}
\maketitle

\section{Solution Representation}
In order to generalize the problem for a vehicle to be able to carry multiple tasks, we decouple each task in two \textit{ConcreteTask} of type \textit{PICKUP} or \textit{DELIVERY}. In the rest of the document, a concrete task is considered to be a task.

\subsection{Variables}
% Describe the variables used in your solution representation %
Our solution representation uses two variables: \textit{firstTasks} and \textit{nextTask}.

\textit{firstTasks} maps a vehicle to its first task. Note that this first task will always be a pick-up task.

\textit{nextTask} maps a task to its next one.

It is important to note that we do not use variables to keep track of vehicles and times anymore. This is because the constraints related to those variables are automatically enforced by our model, as we will see in the next section.

\subsection{Constraints}
% Describe the constraints in your solution representation %
We implement constraints by not constructing forbidden solutions (which violate constraints), except for the weight constraint which is manually checked. 

By generating an initial solution which is valid, and then constructing only valid neighbors, constraints are never violated. As an example, originally, a pick-up task and its associated delivery are handled by the same vehicle, then when assigning the pick-up task to another vehicle, we also move the associated delivery, to keep the constraint satisfied.

Our neighbor generation enforces the following constraints: the constraints 1 to 6 defined in the \textit{Finding the Optimal Delivery Plan} paper, and the following constraints, needed for the problem generalization: 

\begin{itemize}
	\item A pickup should occur before its delivery
	\item A pickup and delivery should be handled by the same vehicle
	\item A task can be picked up only if the total weight carried does not violate the total capacity
\end{itemize}

\subsection{Objective function}
% Describe the function that you optimize %
We decide on solution performance by taking into account its traveling cost, that is, the sum of \textit{distance} $\times$ \textit{cost per km} for each segment along a vehicle's path. 


\section{Stochastic optimization}

\subsection{Initial solution}
% Describe how you generate the initial solution %
Our initial solution attributes the same number of tasks to each vehicle in a round-robin fashion. Tasks are obviously associated to vehicles by pairs \textit{(pickup, delivery)}. Then, all the constraints are respected. 

\subsection{Generating neighbours}
% Describe how you generate neighbors %
We generate neighbors by two different ways, for a random vehicle $v1$ that we pick: 

\begin{itemize}
	\item For each other vehicle, give the first task of $v1$ to the other. 
	\item Exchange each task of $v1$ with every other. 
\end{itemize}

As said above, a neighbor is only generated if it respects the constraints.

\subsection{Stochastic optimization algorithm}
% Describe your stochastic optimization algorithm %
We find the optimal solution as follows: starting with the initial solution as the current one, we generate all the possible neighbors which satisfy the constraints, and keep the best one in term of the objective function. Then, with a certain probability, we take the generated best neighbor as current solution (otherwise we keep the same solution). We repeat this until the cost does not change anymore. 

We do not always take the generated best neighbor in order to avoid getting stuck in a local minimum. 

\section{Results}

\subsection{Experiment 1: Model parameters}
% if your model has parameters, perform an experiment and analyze the results for different parameter values %

\subsubsection{Setting}
% Describe the settings of your experiment: topology, task configuration, number of tasks, number of vehicles, etc. %
% and the parameters you are analyzing %
We use the England topology, with 50 tasks and 4 vehicles, in order to observe the behaviour under the variations of three parameters: \textit{choiceThreshold} responsible for choosing the best next solution or not, \textit{convergenceThreshold} responsible for stopping the optimization after a given number of equal costs, and \textit{iterations} controlling the total number of iterations of the algorithm.

\subsubsection{Observations}
% Describe the experimental results and the conclusions you inferred from these results %
We notice that \textit{choiceThreshold} is mostly responsible for the variance of the optimal solution, i.e., a large threshold implies choosing the next best neighbor most of the time, and then getting the the same minima. A small threshold also implies a high convergence time, since we exploit more the stochastic aspect, by changing the optimization direction randomly.

When increasing the \textit{convergenceThreshold}, we increase the validity of our optimal solution, meaning that such a large threshold will give more chance to the stochastic algorithm to get out of a local minima, then a selected solution is more likely to be the right one. 

As a conclusion, we see that \textit{choiceThreshold} and \textit{convergenceThreshold} are two ways to tune the optimization by restricting the number of iterations and playing with probabilities. However, the only way to make sure we get to the true optimal solution is to use a very large \textit{iterations} number, but still providing the possibility to get out of local minimums with \textit{choiceThreshold}.

\subsection{Experiment 2: Different configurations}
% Run simulations for different configurations of the environment (i.e. different tasks and number of vehicles) %

\subsubsection{Setting}
% Describe the settings of your experiment: topology, task configuration, number of tasks, number of vehicles, etc. %

\subsubsection{Observations}
% Describe the experimental results and the conclusions you inferred from these results %
% Reflect on the fairness of the optimal plans. Observe that optimality requires some vehicles to do more work than others. %
% How does the complexity of your algorithm depend on the number of vehicles and various sizes of the task set? %

\end{document}